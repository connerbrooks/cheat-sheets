\documentclass[10pt,landscape]{article}
\usepackage{multicol}
\usepackage{calc}
\usepackage{ifthen}
\usepackage[landscape]{geometry}
\usepackage{hyperref}

% To make this come out properly in landscape mode, do one of the following
% 1.
%  pdflatex latexsheet.tex
%
% 2.
%  latex latexsheet.tex
%  dvips -P pdf  -t landscape latexsheet.dvi
%  ps2pdf latexsheet.ps


% If you're reading this, be prepared for confusion.  Making this was
% a learning experience for me, and it shows.  Much of the placement
% was hacked in; if you make it better, let me know...


% 2008-04
% Changed page margin code to use the geometry package. Also added code for
% conditional page margins, depending on paper size. Thanks to Uwe Ziegenhagen
% for the suggestions.

% 2006-08
% Made changes based on suggestions from Gene Cooperman. <gene at ccs.neu.edu>


% To Do:
% \listoffigures \listoftables
% \setcounter{secnumdepth}{0}


% This sets page margins to .5 inch if using letter paper, and to 1cm
% if using A4 paper. (This probably isn't strictly necessary.)
% If using another size paper, use default 1cm margins.
\ifthenelse{\lengthtest { \paperwidth = 11in}}
  { \geometry{top=.5in,left=.5in,right=.5in,bottom=.5in} }
  {\ifthenelse{ \lengthtest{ \paperwidth = 297mm}}
    {\geometry{top=1cm,left=1cm,right=1cm,bottom=1cm} }
    {\geometry{top=1cm,left=1cm,right=1cm,bottom=1cm} }
  }

% Turn off header and footer
\pagestyle{empty}
 

% Redefine section commands to use less space
\makeatletter
\renewcommand{\section}{\@startsection{section}{1}{0mm}%
                                {-1ex plus -.5ex minus -.2ex}%
                                {0.5ex plus .2ex}%x
                                {\normalfont\large\bfseries}}
\renewcommand{\subsection}{\@startsection{subsection}{2}{0mm}%
                                {-1explus -.5ex minus -.2ex}%
                                {0.5ex plus .2ex}%
                                {\normalfont\normalsize\bfseries}}
\renewcommand{\subsubsection}{\@startsection{subsubsection}{3}{0mm}%
                                {-1ex plus -.5ex minus -.2ex}%
                                {1ex plus .2ex}%
                                {\normalfont\small\bfseries}}
\makeatother

% Define BibTeX command
\def\BibTeX{{\rm B\kern-.05em{\sc i\kern-.025em b}\kern-.08em
    T\kern-.1667em\lower.7ex\hbox{E}\kern-.125emX}}

% Don't print section numbers
\setcounter{secnumdepth}{0}


\setlength{\parindent}{0pt}
\setlength{\parskip}{0pt plus 0.5ex}


% -----------------------------------------------------------------------

\begin{document}

\raggedright
\footnotesize
\begin{multicols}{3}


% multicol parameters
% These lengths are set only within the two main columns
%\setlength{\columnseprule}{0.25pt}
\setlength{\premulticols}{1pt}
\setlength{\postmulticols}{1pt}
\setlength{\multicolsep}{1pt}
\setlength{\columnsep}{2pt}

\begin{center}
     \Large{\textbf{Big Discrete Cheat Sheet}} \\
\end{center}

\section{3 The Church-Turing Thesis}
\subsection{3.1 Turing Machines}
\subsubsection{Definitions}
A \textbf{\textit{Turing machine}} is a 7-tuple, \\
\((Q, \Sigma, \Gamma, \delta, q_{0}, q_{accept}, q_{reject} )\) \\
\begin{enumerate}
  \item $Q$ is the set of states, 
  \item $\Sigma$ is the input alphabet not containing the \textbf{\textit{blank symbol}} $\sqcup$ 
  \item $\Gamma$ is the tape alphabet, where $\sqcup\ \in\ \Gamma$ and $\Sigma\ \subseteq\ \Gamma$,
  \item $\delta:\ Q \times \Gamma\ \rightarrow Q \times \Gamma \times \{L,\ R\}$
  \item $q_{0}\ \in\ Q$ is the start state,
  \item $q_{accept}\ \in\ Q$ is the accept state, and
  \item $q_{reject}\ \in\ Q$ is the reject state, where $q_{reject} \neq q_{accept}$.
\end{enumerate}

\begin{description}
  \item[Definition 3.5] Call the language \textbf{\textit{Turing-Recognizable}} if some Turing machine recognizes it.
  \item[Definition 3.6] Call a language \textbf{\textit{Turing-decidable}} or simply  \textbf{\textit{decidable}} if some turing machine decides it. \\ 
\end{description}

\subsubsection{Side-notes}
The collection of strings that M accepts is \textbf{\textit{the language of M}}, \\
or \textbf{\textit{the language recognized by M}}, denoted L(M). \\ 

\subsection{3.2 Turing Machine Variants}

\textbf{\textit{Theorems and Corrollaries}}
\begin{itemize}
\item A language is Turing-recognizable if and only if some multitape Turing machine recognizes it.
\item A language is Turing-recognizable if and only if some nondeterministic Turing machine recognizes it.
\item A language is decidable if and only if some nondeterministic Turing machine decides it.
\item A language is Turing-recognizable if and only if some enumerator enumerates it.
\end{itemize}


\section{4 Decidability}
\subsubsection{Other Info}
\begin{itemize}
	\item Every context-free language is decidable	
	\item $E_{TM}$ is undecidable because we can reduce $A_{TM}$ to $E_{TM}$
\end{itemize}

\subsection{Countability}
\begin{itemize}
	\item A set A is \underline{\textbf{countable}} if either it is finite or it has the same size as N.
	\item The set of real numbers is uncountable.
	\item some languages are not Turing-recognizable because there are uncountably many languages yet only countably many turing machines.
	\item For any undecidable language, either it or its complement is not Turing-recognizable.
	\item A language \underline{\textbf{is decidable}} iff it is Turing-recognizable and co-Turing-recognizable.
\end{itemize}

\section{5 Reducibility}
A \textbf{\textit{reduction}} is a way of converting one problem to another problem in such a way that a solution to the second problem can be used to solve the first problem.\\
In terms of computability theory, if A is reducible to B and B is decidable, A also is decidable. Equivalently, if A is undecidable and reducible to B, B is undecidable. \\

\subsubsection{Definitions}
\begin{description}
  \item[Definition 5.5] Let M be a Turing machine and w an input string. An \textbf{\textit{accepting computation history}} for M on w is a sequence of configurations, $C_{1},C_{2},...,C_{l}$ where $C_{1}$ is the start configuration of M, and each $C_{i}$ legally follows from $C_{i-1}$ according to the rules of M. A \textbf{\textit{rejecting computation history}} for M on w is defined similarly, except that $C_{l}$ is a rejecting configuration.
  \item[Definition 5.6] A \textbf{\textit{linear bounded automaton}} is a restricted type of Turing machine wherin the tape head isn't permitted to move off the portion of the tape containing the input. If the machine tries to move its head off either end of the input, the head stays where it is--in the same way that the head will not move off the left-hand end of an ordinary Turing machine's tape.
  \item[Definition 5.20] Language A is \textbf{\textit{mapping reducible}} to language B, written A $\leq_{m}$ B, if ther is a computeable function $f:\ \Sigma^{*} \rightarrow \Sigma^{*}$, where for every w, $w \in A \Longleftrightarrow f(w) \in B$ The function $f$ is called the reduction from A to B.
\end{description}

$HALT_{TM}$ is undecidable because we can reduce $A_{TM}$ to $HALT_{TM}$

\subsubsection{Theorems}
\begin{description}
  \item[5.1] $HALT_{TM}$ is undecidable.
  \item[5.22] If A $\leq_{m}$ B and B is decidable, then A is decidable.
  \item[5.23] If A $\leq_{m}$ B and A is undecidable, then B is undecidable.
  \item[5.30] If $EQ_{TM}$ is neither Turing-recognizable nor co-Turing-recognizable. 
\end{description}

\subsubsection{Corollary}
\begin{description}
  \item[5.29] If A $\leq_{m}$ B and A is not Turing-recognizable, then B is not Turing-recognizable.
\end{description}

\subsection{References}
\subsubsection{Defenitions}
\begin{itemize}
	\item $A_{DFA} = \{\langle B,\omega\rangle | B\  is\ a\ DFA\ that\ accepts\ input\ string\ \omega\}$
	\item $A_{NFA} = \{\langle B,\omega\rangle | B\ is\ an\ NDA\ that\ accepts\ input\ string\ \omega\}$
	\item $A_{REX} = \{\langle R,\omega\rangle | R\ is\ a\ regex\ that\ generates\ string\ \omega\}$
	\item $E_{DFA} = \{\langle A\rangle | A\ is\ a\ DFA\ and\ L(A)\ = \emptyset\}$
	\item $EQ_{DFA} = \{\langle A,B\rangle | A\ and\ B\ are\ DFAs\ and\ L(A) = L(B)\}$
	\item $A_{CFG} = \{\langle G,\omega\rangle | G\ is\ a\ CFG\ that\ generates\ string\ \omega\}$
	\item $E_{CFG} = \{\langle G\rangle | G\ is\ a\ CFG\ and\ L(G) = \emptyset\}$
	\item $EQ_{CFG} = \{\langle G,H\rangle | G\ and\ H\ are\ CFGs\ and\ L(G) = L(H)\}$
	\item $A_{TM} = \{\langle M,\omega\rangle | M\ is\ a\ TM\ and\ M\ accepts\ \omega\}$
  \item $HALT_{TM} = \{\langle M,\omega\rangle | M\ is\ a\ TM\ and\ M\ halts\ on\ input\ \omega\}$
	\item $E_{TM} = \{\langle M\rangle | M\ is\ a\ TM\ and\ L(M) = \emptyset\}$
	\item $REGULAR_{TM} = \{\langle M\rangle | M\ is\ a\ TM\ and\ L(M)\ is\ a\ reg\ lang\}$
	\item $EQ_{TM} = \{\langle M_{1},M_{2}\rangle | M_{1}\ and\ M_{2}\ are\ TMs\ and\ L(M_{1}) = L(M_{2})\}$
	\item $A_{LBA} = \{\langle M,\omega\rangle | M\ is\ an\ LBA\ that\ accepts\ string\ \omega\}$
\end{itemize}

\subsubsection{Info Table}
\begin{tabular}{| l | c |}
    \hline                       
      $A_{DFA}$   & decidable \\ \hline
      $A_{NFA}$   & decidable \\ \hline
      $A_{REX}$   & decidable \\ \hline
      $E_{DFA}$   & decidable \\ \hline
      $EQ_{DFA}$  & decidable \\ \hline
      $A_{CFG}$   & decidable \\ \hline
      $E_{CFG}$   & decidable \\ \hline
      $EQ_{CFG}$  & undecidable \\ \hline
      $A_{TM}$    & undecidable \\ \hline
      $A_{TM}$    & Turing-recognizable \\ \hline
      $\overline{A_{TM}}$ & not Turing Recognizable \\
    \hline  
\end{tabular}

\rule{0.3\linewidth}{0.25pt}
\scriptsize

%Copyright \copyright\ 2015 Conner Brooks 

\href{http://github.com/connerbrooks/cheat-sheets}{http://github.com/connerbrooks/cheat-sheets}

\end{multicols}
\end{document}
