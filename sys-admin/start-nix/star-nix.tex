\documentclass[10pt,landscape]{article}
\usepackage{multicol}
\usepackage{calc}
\usepackage{ifthen}
\usepackage[landscape]{geometry}
\usepackage{hyperref}
\usepackage{listings}

\newcommand{\shellcmd}[1]{\\\indent\indent\texttt{\footnotesize #1}}

% To make this come out properly in landscape mode, do one of the following
% 1.
%  pdflatex latexsheet.tex
%
% 2.
%  latex latexsheet.tex
%  dvips -P pdf  -t landscape latexsheet.dvi
%  ps2pdf latexsheet.ps


% If you're reading this, be prepared for confusion.  Making this was
% a learning experience for me, and it shows.  Much of the placement
% was hacked in; if you make it better, let me know...


% 2008-04
% Changed page margin code to use the geometry package. Also added code for
% conditional page margins, depending on paper size. Thanks to Uwe Ziegenhagen
% for the suggestions.

% 2006-08
% Made changes based on suggestions from Gene Cooperman. <gene at ccs.neu.edu>


% To Do:
% \listoffigures \listoftables
% \setcounter{secnumdepth}{0}


% This sets page margins to .5 inch if using letter paper, and to 1cm
% if using A4 paper. (This probably isn't strictly necessary.)
% If using another size paper, use default 1cm margins.
\ifthenelse{\lengthtest { \paperwidth = 11in}}
  { \geometry{top=.5in,left=.5in,right=.5in,bottom=.5in} }
  {\ifthenelse{ \lengthtest{ \paperwidth = 297mm}}
    {\geometry{top=1cm,left=1cm,right=1cm,bottom=1cm} }
    {\geometry{top=1cm,left=1cm,right=1cm,bottom=1cm} }
  }

% Turn off header and footer
\pagestyle{empty}
 

% Redefine section commands to use less space
\makeatletter
\renewcommand{\section}{\@startsection{section}{1}{0mm}%
                                {-1ex plus -.5ex minus -.2ex}%
                                {0.5ex plus .2ex}%x
                                {\normalfont\large\bfseries}}
\renewcommand{\subsection}{\@startsection{subsection}{2}{0mm}%
                                {-1explus -.5ex minus -.2ex}%
                                {0.5ex plus .2ex}%
                                {\normalfont\normalsize\bfseries}}
\renewcommand{\subsubsection}{\@startsection{subsubsection}{3}{0mm}%
                                {-1ex plus -.5ex minus -.2ex}%
                                {1ex plus .2ex}%
                                {\normalfont\small\bfseries}}
\makeatother

% Define BibTeX command
\def\BibTeX{{\rm B\kern-.05em{\sc i\kern-.025em b}\kern-.08em
    T\kern-.1667em\lower.7ex\hbox{E}\kern-.125emX}}

% Don't print section numbers
\setcounter{secnumdepth}{0}


\setlength{\parindent}{0pt}
\setlength{\parskip}{0pt plus 0.5ex}


% -----------------------------------------------------------------------

\begin{document}

\raggedright
\footnotesize
\begin{multicols}{3}


% multicol parameters
% These lengths are set only within the two main columns
%\setlength{\columnseprule}{0.25pt}
\setlength{\premulticols}{1pt}
\setlength{\postmulticols}{1pt}
\setlength{\multicolsep}{1pt}
\setlength{\columnsep}{2pt}

\begin{center}
     \Large{\textbf{*nix Reference Manual}} \\
\end{center}

\section{PF}
Rules

\subsection{Minimal Rulesets}
\begin{verbatim}
tcp_services = "{ ssh, smtp, domain, 
www, pop3, auth, https, pop3s }"
udp_services = "{ domain }"

block all
pass out proto tcp to port $tcp_services
pass proto udp to port $udp_services

# Allow ssh
pass in inet proto tcp from any to port ssh

# Allow ping
icmp_types = "echoreq"
pass inet proto icmp icmp-type $icmp_types

# Allow loopback
set skip on lo
\end{verbatim}

Reload rule set
\shellcmd{\$ sudo pfctl -f /etc/pf.conf }
\shellcmd{\$ sudo pfctl -vf /etc/pf.conf }



\section{OpenBSD}

\subsection{Set Up PF}
Just enable this
\shellcmd{\$ sudo pfctl -e }

\subsection{Disable users}
DO NOT edit /etc/passwd or /etc/master.password instead
unlock
\shellcmd{\# usermod -U <user>}
lock
\shellcmd{\# usermod -Z <user>}


\section{FreeBSD}
\subsection{Enable PF}
Add the following to /etc/rc.conf:
\shellcmd{pf\_enable="YES"}

Additional options are shown in pfctl(8), they can be passed to PF when it is started.
Add this entry to /etc/rc.conf and specify andy flags 
\shellcmd{pf\_flags=""}

PF will not start if it cannot find its ruleset conf file. The default ruleset is in /etc/pf.conf. 
You can specify a custom conf in /etc/rc.conf.
\shellcmd{pf\_rules="/path/to/pf.conf"}

Example pf.conf file is available in
\shellcmd{/usr/share/examples/pf}

To enable logging in /etc/rc.conf add:
\shellcmd{pflog\_enable="YES"}

To start PF 
\shellcmd{service pf start}
start logging
\shellcmd{service pflog start}


\subsection{General Checks}
Netstat is not the same use sockstat 
\shellcmd{\# sockstat -4l} \\
or you can do what austin do 
\shellcmd{\# netstat -nr} \\


Increase securelevel only do this when were pretty sure not much needs to change!
\begin{verbatim}
sysctl kern.securelevel=2
\end{verbatim}

Show process list
\shellcmd{ps ax }
show network
\shellcmd{netstat -a }

\subsection{Disable users}
DO NOT edit /etc/passwd or /etc/master.password instead
\shellcmd{\# pw lock/unlock <user>}
\shellcmd{\# chsh -s <shell> <user>}

\subsection{Static IP}

To static on boot, in /etc/rc.conf:
\begin{verbatim}
ifconfig_em0="inet 192.168.0.254 netmask 255.255.255.0"
defaultrouter="192.168.0.1"
\end{verbatim}

to do this manually 
\begin{verbatim}
ifconfig em0 inet 192.168.0.254 netmask 255.255.255.0
route delete default; route add default 192.168.0.1
\end{verbatim}


\subsection{ulimit}
Check current limits
\shellcmd{\# ulimit -a }

Set limit on processes
\shellcmd{\# ulimit -u 30 }
may be
\shellcmd{\# ulimit -p 30 }


\subsection{MySql}
Install mysql
\shellcmd{\# pkg install mysql-server }

in /etc/rc.conf add
\begin{verbatim}
mysql_enable="YES"
\end{verbatim}


to start mysql
\shellcmd{\# sh /usr/local/etc/rc.d/mysql-server.sh start}


change root password for mysql
\shellcmd{\# mysql -u root}
If password set
\shellcmd{\# mysql -u root -p}

Anonomous use (WE DONT WANT THIS)
\shellcmd{mysql> SET PASSWORD FOR ''@'localhost' = PASSWORD('newpwd');}
\shellcmd{mysql> SET PASSWORD FOR ''@'host\_name' = PASSWORD('newpwd');} \\

root account
\shellcmd{mysql> SET PASSWORD FOR 'root'@'localhost' = PASSWORD('newpwd');}
\shellcmd{mysql> SET PASSWORD FOR 'root'@'host\_name' = PASSWORD('newpwd');} \\


Show all users for mysql
\shellcmd{mysql> user mysql;}
\shellcmd{mysql> select user,host,password from user;} \\

Reload priveleges
\shellcmd{mysql> flush privileges;}


This can all be done with msqyl\_secure\_installation but make sure!

\rule{0.3\linewidth}{0.25pt}
\scriptsize

%Copyright \copyright\ 2015 Conner Brooks 

\href{http://github.com/connerbrooks/cheat-sheets}{http://github.com/connerbrooks/cheat-sheets}

\end{multicols}
\end{document}
